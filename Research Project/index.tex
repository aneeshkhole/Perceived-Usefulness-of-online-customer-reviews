% Options for packages loaded elsewhere
\PassOptionsToPackage{unicode,linktoc=all}{hyperref}
\PassOptionsToPackage{hyphens}{url}
\PassOptionsToPackage{dvipsnames,svgnames,x11names}{xcolor}
%
\documentclass[
  letterpaper,
  DIV=11,
  numbers=noendperiod]{scrreprt}

\usepackage{amsmath,amssymb}
\usepackage{iftex}
\ifPDFTeX
  \usepackage[T1]{fontenc}
  \usepackage[utf8]{inputenc}
  \usepackage{textcomp} % provide euro and other symbols
\else % if luatex or xetex
  \usepackage{unicode-math}
  \defaultfontfeatures{Scale=MatchLowercase}
  \defaultfontfeatures[\rmfamily]{Ligatures=TeX,Scale=1}
\fi
\usepackage{lmodern}
\ifPDFTeX\else  
    % xetex/luatex font selection
\fi
% Use upquote if available, for straight quotes in verbatim environments
\IfFileExists{upquote.sty}{\usepackage{upquote}}{}
\IfFileExists{microtype.sty}{% use microtype if available
  \usepackage[]{microtype}
  \UseMicrotypeSet[protrusion]{basicmath} % disable protrusion for tt fonts
}{}
\makeatletter
\@ifundefined{KOMAClassName}{% if non-KOMA class
  \IfFileExists{parskip.sty}{%
    \usepackage{parskip}
  }{% else
    \setlength{\parindent}{0pt}
    \setlength{\parskip}{6pt plus 2pt minus 1pt}}
}{% if KOMA class
  \KOMAoptions{parskip=half}}
\makeatother
\usepackage{xcolor}
\usepackage[margin=0.5in]{geometry}
\setlength{\emergencystretch}{3em} % prevent overfull lines
\setcounter{secnumdepth}{4}
% Make \paragraph and \subparagraph free-standing
\ifx\paragraph\undefined\else
  \let\oldparagraph\paragraph
  \renewcommand{\paragraph}[1]{\oldparagraph{#1}\mbox{}}
\fi
\ifx\subparagraph\undefined\else
  \let\oldsubparagraph\subparagraph
  \renewcommand{\subparagraph}[1]{\oldsubparagraph{#1}\mbox{}}
\fi


\providecommand{\tightlist}{%
  \setlength{\itemsep}{0pt}\setlength{\parskip}{0pt}}\usepackage{longtable,booktabs,array}
\usepackage{calc} % for calculating minipage widths
% Correct order of tables after \paragraph or \subparagraph
\usepackage{etoolbox}
\makeatletter
\patchcmd\longtable{\par}{\if@noskipsec\mbox{}\fi\par}{}{}
\makeatother
% Allow footnotes in longtable head/foot
\IfFileExists{footnotehyper.sty}{\usepackage{footnotehyper}}{\usepackage{footnote}}
\makesavenoteenv{longtable}
\usepackage{graphicx}
\makeatletter
\def\maxwidth{\ifdim\Gin@nat@width>\linewidth\linewidth\else\Gin@nat@width\fi}
\def\maxheight{\ifdim\Gin@nat@height>\textheight\textheight\else\Gin@nat@height\fi}
\makeatother
% Scale images if necessary, so that they will not overflow the page
% margins by default, and it is still possible to overwrite the defaults
% using explicit options in \includegraphics[width, height, ...]{}
\setkeys{Gin}{width=\maxwidth,height=\maxheight,keepaspectratio}
% Set default figure placement to htbp
\makeatletter
\def\fps@figure{htbp}
\makeatother
% definitions for citeproc citations
\NewDocumentCommand\citeproctext{}{}
\NewDocumentCommand\citeproc{mm}{%
  \begingroup\def\citeproctext{#2}\cite{#1}\endgroup}
\makeatletter
 % allow citations to break across lines
 \let\@cite@ofmt\@firstofone
 % avoid brackets around text for \cite:
 \def\@biblabel#1{}
 \def\@cite#1#2{{#1\if@tempswa , #2\fi}}
\makeatother
\newlength{\cslhangindent}
\setlength{\cslhangindent}{1.5em}
\newlength{\csllabelwidth}
\setlength{\csllabelwidth}{3em}
\newenvironment{CSLReferences}[2] % #1 hanging-indent, #2 entry-spacing
 {\begin{list}{}{%
  \setlength{\itemindent}{0pt}
  \setlength{\leftmargin}{0pt}
  \setlength{\parsep}{0pt}
  % turn on hanging indent if param 1 is 1
  \ifodd #1
   \setlength{\leftmargin}{\cslhangindent}
   \setlength{\itemindent}{-1\cslhangindent}
  \fi
  % set entry spacing
  \setlength{\itemsep}{#2\baselineskip}}}
 {\end{list}}
\usepackage{calc}
\newcommand{\CSLBlock}[1]{\hfill\break\parbox[t]{\linewidth}{\strut\ignorespaces#1\strut}}
\newcommand{\CSLLeftMargin}[1]{\parbox[t]{\csllabelwidth}{\strut#1\strut}}
\newcommand{\CSLRightInline}[1]{\parbox[t]{\linewidth - \csllabelwidth}{\strut#1\strut}}
\newcommand{\CSLIndent}[1]{\hspace{\cslhangindent}#1}

\KOMAoption{captions}{tableheading}
\usepackage{float}
\usepackage{booktabs, caption, longtable, colortbl, array}
\floatplacement{table}{H}
\floatplacement{image}{H}
\makeatletter
\@ifpackageloaded{bookmark}{}{\usepackage{bookmark}}
\makeatother
\makeatletter
\@ifpackageloaded{caption}{}{\usepackage{caption}}
\AtBeginDocument{%
\ifdefined\contentsname
  \renewcommand*\contentsname{Table of contents}
\else
  \newcommand\contentsname{Table of contents}
\fi
\ifdefined\listfigurename
  \renewcommand*\listfigurename{List of Figures}
\else
  \newcommand\listfigurename{List of Figures}
\fi
\ifdefined\listtablename
  \renewcommand*\listtablename{List of Tables}
\else
  \newcommand\listtablename{List of Tables}
\fi
\ifdefined\figurename
  \renewcommand*\figurename{Figure}
\else
  \newcommand\figurename{Figure}
\fi
\ifdefined\tablename
  \renewcommand*\tablename{Table}
\else
  \newcommand\tablename{Table}
\fi
}
\@ifpackageloaded{float}{}{\usepackage{float}}
\floatstyle{ruled}
\@ifundefined{c@chapter}{\newfloat{codelisting}{h}{lop}}{\newfloat{codelisting}{h}{lop}[chapter]}
\floatname{codelisting}{Listing}
\newcommand*\listoflistings{\listof{codelisting}{List of Listings}}
\makeatother
\makeatletter
\makeatother
\makeatletter
\@ifpackageloaded{caption}{}{\usepackage{caption}}
\@ifpackageloaded{subcaption}{}{\usepackage{subcaption}}
\makeatother
\ifLuaTeX
  \usepackage{selnolig}  % disable illegal ligatures
\fi
\usepackage{bookmark}

\IfFileExists{xurl.sty}{\usepackage{xurl}}{} % add URL line breaks if available
\urlstyle{same} % disable monospaced font for URLs
\hypersetup{
  pdftitle={CSCI 5302 - Final Project},
  pdfauthor={Patrick Connelly; Aneesh Khole; Uttara Ketkar},
  colorlinks=true,
  linkcolor={blue},
  filecolor={Maroon},
  citecolor={Blue},
  urlcolor={Blue},
  pdfcreator={LaTeX via pandoc}}

\title{CSCI 5302 - Final Project}
\author{Patrick Connelly \and Aneesh Khole \and Uttara Ketkar}
\date{2024-02-20}

\begin{document}
\maketitle

\renewcommand*\contentsname{Table of contents}
{
\hypersetup{linkcolor=}
\setcounter{tocdepth}{2}
\tableofcontents
}
\bookmarksetup{startatroot}

\chapter{Introduction}\label{sec-intro}

\section{Concept and Motivation}\label{concept-and-motivation}

Customers, when searching for products with specific features and
aspects, need sufficient information to make a decision as to whether to
procure a specific product. \emph{(Information on search products
vs.~experience vs.~mixed).} When a product is more in the directon of
experience vs.~search-based, other customers' experiences can shed light
on its features and return on investment than information directly from
the vendor can. Having reviews from reliable sources with sufficiently
detailed information can enable greater confidence in a purchase,
improved customer satisfaction, and smooth the process of ecommerce for
customers.

We seek to expound upon the research of
(\citeproc{ref-percUse}{\textbf{percUse?}}) to explore additional
recommended research areas to improve upon and increase the general
applicability of the model.

\section{What / Where}\label{what-where}

(\citeproc{ref-percUse}{\textbf{percUse?}}) provided the following areas
for recommended additional research at the conclusion of their paper:

\begin{enumerate}
\def\labelenumi{\arabic{enumi}.}
\item
  Expand the number of products beyond 3 items (one search, one
  experience, one mixed) to better generalize the model.
\item
  Explore customer or reviewer metadata for classifying reviewer types
  to enhance model performance.
\end{enumerate}

We seek to explore Item \#1 and \#2 above, and to explore the
possibility of assessing a scale for products to determine the extent to
which they are a search or experience-based product. We've explored work
from other research teams to identify potential methods we can leverage
to pursue these ends.

\begin{itemize}
\item
  Determine the polarity of a customer review by employing a classifier
  such as Naive Bayes.
\item
  Use Kansei engineering approaches to convert unstructured
  product-related texts into feature--affective opinions.
\item
  Attempt to assess the reliability of a customer's review based on
  star-rating and a `sentiment score' of their textual feedback.
\end{itemize}

Exploring combinations of these research methods, we will pursue
potential improvements on the models outlined in
(\citeproc{ref-percUse}{\textbf{percUse?}}). We will examine additional
products and product types between multiple e-commerce websites
(BestBuy, Target, Amazon).

\section{Why It Matters}\label{why-it-matters}

Feedback from customers is beneficial, but it is not always ordered by
the most informative or beneficial feedback first. Certain features of
data such as\ldots{} can impact the usefulness of the feedback on a
customer-by-customer basis. Level of detail, star-rating, and number of
votes that support the review as being useful to a customer can all help
determine its usefulness to other customers. Were e-commerce

Examining additional product types can enable the generalization of the
authors' methodology to other products. Furthermore, the exploration of
a sliding scale for search vs.~experience-based products can further
support generalization and business goals. Producing a reliable scale
and methods for classifying a products' degree of being
experienced-based can inform vendors on:

\begin{itemize}
\item
  How to best sort product reviews
\item
  Examine what are the most helpful reviews to know the performance of
  the product alongside customer experience and sentiment
\item
  Adjust the product, its marketing, or future production based upon
  market efficacy.
\item
  Understanding the emotions a customer wants to express through a
  review is crucial as it will affect the ``recommendation score'' of
  that particular product or a different one from a similar category.

  \begin{itemize}
  \item
    To contribute in determining this recommendation score, we can use a
    probabilistic machine learning algorithm like Naive Bayes to
    determine the polarity (positive, negative, or neutral) of customer
    reviews.
  \item
    Typically used for amending product design, Kansei Engineering can
    be used to incorporate human emotional responses into evaluation of
    a customer review.
  \end{itemize}
\item
  Determine which customer is trustworthy, meaning who has actually
  purchased the product versus a customer who gave a false review. Based
  on the `customer reputation score', our aim is to classify customers
  into groups to judge reviewer reliability. This has two main aspects:

  \begin{itemize}
  \item
    Star-rating score which is a discrete scale that tells the
    inclination of a customer.
  \item
    Text review `sentiment score' using NLP that explains customer
    opinions based on words.
  \end{itemize}
\end{itemize}

\section{Literature Survey}\label{literature-survey}

\begin{itemize}
\item
  \textbf{Additional commentary on original paper here}
\item
  \textbf{Paper(s) on product classification (search-experience-mixed)}
\item
  \textbf{Paper(s) on user/consumer/reviewer classification}
\item
  \textbf{All papers you've found, provide a summary of what they did
  and any key results}

  \begin{itemize}
  \item
    Hu, W., Gong, Z., \& Guo, J. (2010). Mining Product Features from
    Online Reviews. 2010 IEEE 7th International Conference on E-Business
    Engineering. doi:10.1109/icebe.2010.51

    \begin{itemize}
    \tightlist
    \item
      The proposed system employs a two-step process for opinion mining:
      identifying opinion sentences using a SentiWordNet-based algorithm
      and extracting product features from all reviews in the database.
      This feature extraction function focuses on identifying commonly
      expressed positive or negative opinions before extracting explicit
      and implicit product features.
    \end{itemize}
  \item
    Rajeev, P. V., \& Rekha, V. S. (2015). Recommending products to
    customers using opinion mining of online product reviews and
    features. 2015 International Conference on Circuits, Power and
    Computing Technologies {[}ICCPCT-2015{]}.
    doi:10.1109/iccpct.2015.7159433

    \begin{itemize}
    \tightlist
    \item
      This paper presents techniques like Opinion mining, feature
      extraction and Naives Bayes classification for review polarity
      determination. The authors suggest performing both Objective and
      Subjective analysis of features by considering qualitative and
      quantitative features of the data respectively.
    \end{itemize}
  \item
    Wang, W. M., Li, Z., Tian, Z. G., Wang, J. W., \& Cheng, M. N.
    (2018). Extracting and summarizing affective features and responses
    from online product descriptions and reviews: A Kansei text mining
    approach. Engineering Applications of Artificial Intelligence, 73,
    149--162. doi:10.1016/j.engappai.2018.05.005

    \begin{itemize}
    \tightlist
    \item
      Authors have proposed a solution by implementing Kansei
      engineering and text mining simultaneously which will help
      customers in decision making process. It helps to categorize
      reviews into multiple sections and perform text mining by NLP
      techniques like Sentence segmentation, Tokenization, POS tagging
    \end{itemize}
  \end{itemize}
\end{itemize}

\section{Research Questions}\label{research-questions}

\begin{itemize}
\item
  Can the model from (\citeproc{ref-percUse}{\textbf{percUse?}}) be
  generalized with

  \begin{itemize}
  \item
    larger volume of products and product types from which to mine data?
  \item
    a sliding scalar multiplier representing the degree to which a
    product is a ``search'' (0) or ``experience'' (1) product?
  \item
    Adding modifiers to review content based upon:

    \begin{itemize}
    \tightlist
    \item
      Customer / Reviewer Reliability?
    \end{itemize}
  \end{itemize}
\item
  Can the polarity of reviews be judged accurately by using a Naive
  Bayes classification model?

  \begin{itemize}
  \tightlist
  \item
    What is the impact of different feature extraction methods (e.g.,
    bag-of-words, TF-IDF) on the performance of Naive Bayes
    classification model?
  \end{itemize}
\item
  Can products be classified on their degree of being search or
  experience based by examining product variables such as:

  \begin{itemize}
  \item
    Degree of specificity in the product description?
  \item
    Whether the product is offered in brand-new condition only, or
    offered as new, used, or refurbished?
  \item
    Which of the 5 senses the product engages?
  \item
    Item rarity (limited production or unique items vs.~bulk-produced
    items)?
  \end{itemize}
\item
  Can newer natrual language processing libraries provide a better fit
  for (\citeproc{ref-percUse}{\textbf{percUse?}}) 's Review Content
  metrics?
\item
  How does sentiment in customer reviews correlate with customer
  satisfaction metrics or sales figures for a particular product?
\item
  Can we categorize customer reviews based on customer experience and
  sentiment?
\item
  Do specific product star ratings tend to incite more reviews, and if
  so, how does this impact the overall reputation measurement?
\item
  Are specific quality descriptors in text-based reviews (e.g.,
  `enthusiastic', `disappointed') strongly associated with certain
  rating levels, and how does this association affect product
  reputation?
\end{itemize}

\section{Goals / Definition of
Success}\label{goals-definition-of-success}

\begin{itemize}
\item
  Replicate similar results to
  (\citeproc{ref-percUse}{\textbf{percUse?}}) with similar product types
\item
  Expound upon (\citeproc{ref-percUse}{\textbf{percUse?}}) with
  additional products, including:

  \begin{enumerate}
  \def\labelenumi{\alph{enumi}.}
  \item
    Original products from (paper): Digital Music, Video Game, and
    Grocery Item
  \item
    Additional products (Amazon and Target): Furniture Items, Clothing
    Items, Home Appliances, Books, Cosmetics, Cleaning supplies
  \item
    Additional Proucts (Amazon, Target, BestBuy): Electronics
  \item
    Verify goodness of fit of original model
  \end{enumerate}
\item
  Determing best metrics and/or modifiers for Review Content and
  Customer Reliability
\item
  Achieving similar or better fit than original paper's modeling;
  extrapolate to other product types.
\end{itemize}

\begin{itemize}
\item
  Determining strength of correlation metrics (support, confidence,
  lift) between Naive Bayes' classifier for review polarity

  \begin{itemize}
  \item
    Integrate with the model and test if Naive Bayes shows strong
    correlation metrics.
  \item
    Compare and contrast the model with and without incorporation.
  \end{itemize}
\item
\end{itemize}

\bookmarksetup{startatroot}

\chapter{Data}\label{sec-data}

Here's our data section.

\section{Some items to research}\label{some-items-to-research}

\begin{itemize}
\item
  \href{https://en.wikipedia.org/wiki/SEC_classification_of_goods_and_services}{Wikipedia}
\item
\end{itemize}

\section{Collection}\label{collection}

In collecting our data, in order to adhere to the model implemented by
\emph{(original paper here)}, we require the following data points, at a
minimum:

\begin{longtable}[]{@{}ll@{}}
\caption{}\label{T_68bd3}\tabularnewline
\toprule\noalign{}
Variable & Data Type \\
\midrule\noalign{}
\endfirsthead
\toprule\noalign{}
Variable & Data Type \\
\midrule\noalign{}
\endhead
\bottomrule\noalign{}
\endlastfoot
Star Rating & float \\
Review Content & string \\
Useful Votes & integer \\
\end{longtable}

To classify products\ldots we need metadata of the product itself. We
are seeking to use the following products, and for each, we'll collect
the following data:

\begin{itemize}
\item
  Original Products: (update to table later)

  \begin{itemize}
  \item
    Video Games (rating I.e. pg, pg-13, R, but from MSRB ratings).
  \item
    Digital/Physical Music (duration, music style / genre, others?)
  \item
    Grocery Products (calories per serving, special markers
    {[}gluten-free, fat-free, vegan{]}, ) - \emph{potential hypothesis
    that grocery products with special markings may be more
    experience-based than search-based.}
  \end{itemize}
\end{itemize}

Additional Products

\begin{itemize}
\item
  Clothing - likely a mixed product, potentially ranging from
  search-based for plain t-shirts to experience-based for high fashion
  products

  \begin{itemize}
  \tightlist
  \item
    material?
  \end{itemize}
\item
  Laptops / Computers

  \begin{itemize}
  \item
    Likely a mix between search and experience
  \item
    RAM, CPU, Storage, OS Included, Graphics Type, \ldots{}
  \end{itemize}
\item
  Something else?
\end{itemize}

\section{Preparation, Standardization, and
Cleaning}\label{preparation-standardization-and-cleaning}

\begin{itemize}
\item
  Gathering from Amazon (All Products)
\item
  Gathering from BestBuy (Electronic Products, no grocery or clothing)
\item
  Gathering from Target (All products)
\end{itemize}

\section{Visualization}\label{visualization}

\bookmarksetup{startatroot}

\chapter{Models Implemented}\label{sec-models-implemented}

Here's where we'll list out our models

\section{Examined Models from Literature
Research}\label{examined-models-from-literature-research}

\begin{itemize}
\item
  Perceived Usefulness of Product Reviews

  \begin{itemize}
  \tightlist
  \item
    Modeling for search vs.~experience goods as ``mitigations'' on
    reviews, examining their impact on perceived review usefulness
  \end{itemize}
\item
  Wealth of Data Paper on Customer Evaulation

  \begin{itemize}
  \item
    Examination of `Vine' customers from Amazon
  \item
    Don't think that BestBuy or Target have the same.
  \item
    Scope of this would be limited to Amazon, combining research from
    Perceived Usefulness paper (i.e.~their modeling) with a customer
    ``trustworthiness'' score for Amazon Vine customers

    \begin{itemize}
    \item
      May have challenges identifying products with Vine customers
    \item
      Should also have a look at ``verified purchase'' customers, too.
    \end{itemize}
  \item
    Weaknesses (from authors)

    \begin{itemize}
    \item
      Their NLP instance wasn't well trained for fully accurate
    \item
      Their weighting system was manually (i.e.~arbitrarily) set as
      opposed to having a labeled dataset
    \item
      Can we fix these problems? If so - what data is needed?
    \end{itemize}
  \end{itemize}
\end{itemize}

\section{Potential Model for Search vs.~Experience Weight (scale of
0-1)}\label{potential-model-for-search-vs.-experience-weight-scale-of-0-1}

\subsection{Jaccard Similarity
Measure}\label{jaccard-similarity-measure}

https://www.sciencedirect.com/science/article/pii/S1567422318300450

\begin{itemize}
\item
  Can be used to produce a similarity between two items on a scale of 0
  to 1
\item
  May be able to use this for evaluation of a novel item against a
  pure-search good vs.~a pure-experience good.

  \begin{itemize}
  \item
    May require additional calculation / computation between these two
    values - maybe its arithmetic or harmonic mean between search and
    experience
  \item
    Using that, we could potentially produce a weight.
  \item
    Authors models

    \begin{itemize}
    \item
      Model 1:
      \(\text{Perceived Usefulness} = \beta_0 + \beta_1 \cdot \text{Review Content} + \beta_2 \cdot \text{Review Length} + \beta_3\cdot \text{Star Rating} + \beta_4 \cdot \text{Total Votes Received} + \epsilon_1\)
    \item
      Model 2 (product type as a moderator):
      \(\text{Perceived Usefulness} = \beta_0 + \beta_1 \cdot \text{Review Content} + \beta_2 \cdot \text{Review Length} + \beta_3\cdot \text{Star Rating} + \beta_4 \cdot \text{Total Votes Received} + \beta_5 \cdot \text{Digital Music} + \beta_6 \cdot \text{Video Game} + \beta_7 \cdot \text{Review Content}\cdot \text{ Digital Music} + \beta_8\cdot\text{Review Content}\cdot\text{ Video Game} + \epsilon_2\)
    \item
      Our proposition \#1 (may require some modification) \[
      \text{Perceived Usefulness} = \beta_0 + \beta_1 \cdot \text{Review Content} + \beta_2 \cdot \text{Review Length} + \beta_3\cdot \text{Star Rating} + \beta_4 \cdot \text{Total Votes Received} + \gamma\cdot\text{Review Content}
      \]
    \end{itemize}
  \end{itemize}
\end{itemize}

or

\[
\text{Perceived Usefulness} = \beta_0 + \gamma\cdot\beta_1\cdot\text{Review Content} + \beta_2 \cdot \text{Review Length} + \beta_3\cdot \text{Star Rating} + \beta_4 \cdot \text{Total Votes Received}
\]

\begin{itemize}
\item
  Where \(\gamma\) is the Jaccard Similarity Score between a given
  product and elements we are identifying as ``pure'' experience and
  ``pure'' search good.
\item
  Operationalizaton of Jaccard Similarity score Variable (i.e.~inputs)

  \begin{itemize}
  \item
    Search good: those with attributes that can be evaluated prior to
    purchase or consumption. Consumers rely on prior experience, direct
    product inspection and other information search activities to locate
    information that assists in the evaluation process. Most products
    fall into the search goods category (e.g.~clothing, office
    stationery, home furnishings).
  \item
    Number of measurement specifications?
  \item
    comment/review information?
  \item
    What else could we gather that could be considered a ``universal''
    tangible or intangible feature from a product online? They need to
    be applicable to both search and experience goods.
  \end{itemize}
\item
  Experience good: those that can be accurately evaluated only after the
  product has been purchased and experienced. Many personal services
  fall into this category (e.g.~restaurant, hairdresser, beauty salon,
  theme park, travel, holiday).

  \begin{itemize}
  \item
    subjective descriptiveness vs.~
  \item
    comment/review information?
  \end{itemize}
\item
  Are there other methods aside from Jaccard Similarity?
\end{itemize}

\section{Model Comparison}\label{model-comparison}

\bookmarksetup{startatroot}

\chapter{Conclusions}\label{sec-conclusions}

Here's our conclusions section

\bookmarksetup{startatroot}

\chapter*{References}\label{references}
\addcontentsline{toc}{chapter}{References}

\markboth{References}{References}

\phantomsection\label{refs}
\begin{CSLReferences}{0}{1}
\end{CSLReferences}



\end{document}
